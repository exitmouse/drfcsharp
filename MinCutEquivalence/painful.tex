\documentclass[12pt]{article}
\usepackage{amsmath, amsthm, amssymb}
\usepackage{fullpage, paralist}
\usepackage{graphicx}      % specify `dvips' driver
\usepackage{subfig}
\usepackage[all]{xy}

\CompileMatrices

\newcommand{\Z}{{\mathbb Z}}
\newcommand{\Q}{{\mathbb Q}}
\newcommand{\R}{{\mathbb R}}
\newcommand{\C}{{\mathbb C}}

\newcommand{\A}{{\mathcal A}}
\newcommand{\I}{{\mathcal I}}
\newcommand{\B}{{\mathcal B}}
\newcommand{\cC}{{\mathcal C}}
\newcommand{\F}{{\mathbb F}}

\newcommand{\tmin}{\text{min}}
\newcommand{\tmax}{\text{max}}
\newcommand{\Spl}{\text{Spl}}

\newcommand{\cov}{\textbf{cov}}
\newcommand{\var}{\textbf{var}}
\newcommand{\E}{\mathbb{E}}
\newcommand{\p}{\text{P}}

\newcommand{\la}{\langle}
\newcommand{\ra}{\rangle}
\newtheorem{thm}{Theorem}[section]

\theoremstyle{remark}
\newtheorem*{nnremark}{Remark}

\begin{document}
\title{Computer Science 74}
\author{Jesse Selover and Dan Denton}
\markright{Jesse Selover and Dan Denton}
\maketitle


In the DRF model, we have
\[
\p(\vec{x}\mid\vec{y}) = \frac{1}{Z} \exp\left(\sum_{i\in S}
  \log(\sigma(x_i \vec{w}^{\top}\vec{h}_i(\vec{y}))) + \sum_{i \in S}
  \sum_{j \in \mathcal{N}_i}
  x_ix_j\vec{v}^{\top}\vec{\mu}_{ij}(\vec{y}) \right).
\]
Each $x_i \in \{-1,1\}$. Greig et al. used 

Following Greig et al., we write the log apart from an additive
constant:
\[
L(\vec{x} \mid \vec{y}) = \sum_{i \in S} \log(\sigma(x_i
\vec{w}^{\top}\vec{h}_i(\vec{y}))) + \sum_{i \in S}\sum_{j \in \mathcal{N}_i}
x_ix_j\vec{v}^{\top}\vec{\mu}_{ij}(\vec{y}).
\]
Rewriting,
\[
L(\vec{x} \mid \vec{y}) = \sum_{i \in S}
\log(\sigma((\vec{w}^{\top}\vec{h}_i(\vec{y})) x_i)) + \sum_{i \in S}\sum_{j \in \mathcal{N}_i}
(\vec{v}^{\top}\vec{\mu}_{ij}(\vec{y}))x_ix_j.
\]
Now let's say we made a graph with the $\lambda_i =
\log(\sigma((\vec{w}^{\top}\vec{h}_i(\vec{y})) x_i))$ and we'll figure
out the internode edges later.
A cut on the graph would have capacity

%We want
%\[
%\arg\max_\theta \sum_{m=1}^M \sum_{i \in S} \left[
%  \log(\sigma(x_i\vec{w}^\top \vec{h}_i(\vec{y}))) + \sum_{j \in
%    \mathcal{N}_i} x_i x_j \vec{v}^\top \vec{\mu}_{ij}(\vec{y}) -
%  \log(z_i) \right] - \frac{1}{2\tau^2}\vec{v}^\top\vec{v}.
%\]
%We have
%\[
%\frac{\partial}{\partial w_k} \text{that} = \sum_{m=1}^M \sum_{i \in
%  S} \left[ x_i(\vec{h}_i(\vec{y}))_k(1-\sigma(x_i \vec{w}^\top
%  \vec{h}_i(\vec{y}))) - \frac{\frac{\partial}{\partial w_k} z_i}{z_i}
%\right]
%\]
%Computing the derivative of $z_i$,
%\[
%\frac{\partial}{\partial w_k} z_i = \sum_{x_i \in \{-1,1\}}
%\exp\left(\log(\sigma(x_i\vec{w}^\top \vec{h}_i(\vec{y}))) +
%  \sum_{j \in \mathcal{N}_i} x_i x_j \vec{v}^\top
%  \vec{\mu}_{ij}(\vec{y})\right)\left(1-\sigma(x_i\vec{w}^\top\vec{h}_i(\vec{y}))\right)
%  \left(x_i(\vec{h}_i(\vec{y}))_k\right)
%\]
%Now for $v_k$
%\[
%\frac{\partial}{\partial v_k} = \sum_{m=1}^M \sum_{i \in S} \left[
%  \sum_{j \in \mathcal{N}_i} x_i x_j (\vec{\mu}_{ij}(\vec{y}))_k -
%  \frac{\frac{\partial}{\partial v_k} z_i}{z_i} \right] - \frac{v_k}{\tau^2}
%\]
%where the partial derivative of $z_i$ is
%\[
%\frac{\partial}{\partial v_k} z_i = \sum_{x_i \in \{-1,1\}}
%\exp\left(\log(\sigma(x_i\vec{w}^\top \vec{h}_i(\vec{y}))) +
%  \sum_{j \in \mathcal{N}_i} x_i x_j \vec{v}^\top
%  \vec{\mu}_{ij}(\vec{y})\right)\sum_{j \in \mathcal{N}_i} x_i x_j(\vec{\mu}_{ij}(\vec{y}))_k
%\]
\end{document}
